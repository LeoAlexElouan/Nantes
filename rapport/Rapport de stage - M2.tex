\documentclass{article}
\usepackage{graphicx} % Required for inserting images
\usepackage[a4paper, lmargin = 1cm, rmargin =1cm, tmargin = 1cm, bmargin = 2cm]{geometry}


\usepackage{tikz}
\usetikzlibrary{cd}
\usepackage{quiver}

\usepackage[sorting=none]{biblatex}
\addbibresource{reference.bib}


\usepackage{mathrsfs}
\usepackage{amsmath}
\usepackage{amssymb}
\usepackage{amsthm}
\usepackage{stmaryrd }
\usepackage[T1]{fontenc}
\usepackage{euscript}



\usepackage{adjustbox}

\usepackage{color}


\usepackage{semantic}

\newtheorem{theorem}{Theorem}[section]
\newtheorem{lemma}[theorem]{Lemma}
\newtheorem{definition}[theorem]{Definition}

\DeclareMathOperator{\colim}{colim}
\DeclareMathOperator{\Prop}{\mathsf{Prop}}
\DeclareMathOperator{\SProp}{\mathsf{SProp}}
\DeclareMathOperator{\Type}{\mathsf{Type}}
\DeclareMathOperator{\ask}{\mathsf{ask}}
\DeclareMathOperator{\bind}{\mathsf{bind}}
\DeclareMathOperator{\Inductive}{\mathsf{Inductive}}
\DeclareMathOperator{\ret}{\mathsf{ret}}
\DeclareMathOperator{\rec}{rec}
\DeclareMathOperator{\ctx}{\mathsf{ctx}}
\DeclareMathOperator{\Typ}{\mathsf{Typ}}
\DeclareMathOperator{\trm}{\mathsf{Trm}}

\newcommand{\ovarr}{\overrightarrow}
\newcommand{\ovl}{\overline}
\newcommand{\omlim}{A_0+\sum\limits_{i\geq 1}\ovl A_i}

\newcommand{\0}{\mathbf{0}}
\newcommand{\1}{\mathbf{1}}
\newcommand{\nat}{\mathbf{N}}
\newcommand{\C}{\mathcal{C}}
\newcommand{\D}{\mathcal{D}}
\newcommand{\SBS}{\mathbf{SBS}}

\newcommand{\Gram}{\mathsf{G}}
\newcommand{\mono}{\rightarrowtail}
\newcommand{\omop}{\omega^{op}}
\newcommand{\omopch}{\Theta}
\newcommand{\omch}{\Omega}
\newcommand{\Dist}{\mathsf{Dist}}
\newcommand{\gd}{\dot {\mathbf{Fun}}}
\newcommand{\colomf}[1]{\varinjlim\circ\omch_{#1}^{(-)}}
\newcommand{\colom}[2]{\varinjlim(\omch_{#1}^{#2})}
\newcommand{\lcontext}{\Gamma}
\newcommand{\rcontext}{\Delta}
\newcommand{\lvar}{\gamma}
\newcommand{\rvar}{\delta}

\newcommand{\tcontext}{\Gamma}
\newcommand{\tvar}{\gamma}
\newcommand{\diag}{\Delta}
\newcommand{\DThunk}{\mathsf{Dist}}

\newcommand{\Unit}{\mathsf{Unit}}
\newcommand{\Bool}{\mathsf{Bool}}
\newcommand{\Real}{\mathsf{Real}}
\newcommand{\Nat}{\mathsf{Nat}}
\newcommand{\List}{\textsf{List}}
\newcommand{\Stream}{\textsf{Stream}}

\newcommand{\Wf}{\textsf{well-formed}}
\newcommand{\slvl}{\mathbf{s}}
\newcommand{\llvl}{\mathbf{l}}


\mathlig{<-}{\leftarrow}
\mathlig{->}{\to}
\mathlig{|->}{\mapsto}
\mathlig{=>}{\Rightarrow}
\mathlig{|-}{\vdash}
\mathlig{[|}{\llbracket}
\mathlig{|]}{\rrbracket}

\mathlig{==}{\equiv}

\reservestyle{\command}{\mathsf}
\command{inl,inr,wrap,roll,rec,corec,as,match,return,let,in,sample,on,do}


% Distributions in programming language
\newcommand{\Uniform}{\mathsf{uniform}}


\title{Internship report M2\\ Parisian Master of Research in Computer Science Level 
2\vspace{1cm}\\  Non-singleton Elimination\\}
\author{\textbf{Soudant L\'eo}\\ fourth year student at ENS Paris Saclay}
\date{2025}


\begin{document}


\begin{titlepage}
    \maketitle
    \centering
    Supervisor
    
    \textbf{Pierre-Merie P\'edrot}, chargé de recherche at Inria Rennes-Bretagne-Atlantique, 

\end{titlepage}

\begin{abstract}
\end{abstract}
\tableofcontents
\newpage


\section{Introduction}




\subsection{The outline}

\section{Sheaves}
\subsection{Topoï and sheaves in topoï}


Proofs, results and details for this section can often be found in \emph{Sheaves in geometry and logic: A first introduction to topos theory} by Saunders Maclane and Ieke Moerdijk \cite{maclane2012sheaves}.

\begin{definition}[Subobject]
    In a category, a \emph{subobject} of $X$ is an equivalence class of monomorphism $m : A \mono X$, where the equivalence comes from the preorder where $ m : A \mono X $ is smaller than $ m' : A' \mono X$ when there is a map $f : A -> A'$ with $ m' \circ f = m$.
\end{definition}

We deduce a presheaf $\mathbf{Sub}$ where $\mathbf{Sub}(X)$ is the the set of subobjects of $X$, and $\mathbf{Sub}(f) : \mathbf{Sub}(Y) -> \mathbf{Sub}(X)$ sends $m : A -> Y$ to its pullback by $f : X -> Y $. 

\begin{definition}[Topos]
    A \emph{topos} is a cartesian closed category with all finite limits and a suboject \emph{classifier} $\Omega$ and an isomorphism $\mathbf{Sub}(X)\cong \mathbf{Hom}(X, \Omega)$ natural in $X$.
\end{definition}

We note that topoï also have finite colimits.

A topos serves to give models of intuitionistic logic in classical mathematical language. It has an internal logic which is higher order.

For example, $\mathbf{Set}$ is a topos, and given a topos $\mathcal E$, $\mathcal{E}/X$, the category of maps with codomain $X$ and commuting triangles, as well as $\mathcal{E}^{\mathbf{C}^{op}}$, the category of contravariant functors from a small category $\mathbf{C}$ and natural transformation, are all topoï. In particular categories of presheaves are topoï, and correspond to Kripke models. 

The subobject classifier $\Omega$ is equipped with an internal meet-semilattice structure inherited from the meet-semilattice structure on each $\mathbf{Sub}(X)$, which is natural in $X$.

\begin{definition}[Lawvere-Tierney topology]
    A \emph{Lawvere-Tierney topology} is a left exact idempotent monad $j$ on the internal meet-semilattice on $\Omega$.
\begin{itemize}
    \item $ id_\Omega \leq j $,
    \item $ j\circ j \leq  j$
    \item $ j \circ \wedge = \wedge \circ j\times j$
\end{itemize}
\end{definition}

From a topology $j$ we extract a closure operator $J_X$ of $\mathbf{Sub}(X)$ for any $X$.
\begin{definition}[Dense subobject]
    A suboject $U$ of $X$ is dense if $J_XU  = X$
\end{definition}

A topology can be lifted to a left exact idempotent monad on the entirety of the topos, the sheafification monad.

\begin{definition}[$j$-Sheaf in topos]
    An object $F$ is a $j$-\emph{sheaf} in a topos if for any dense subobject $U$ of any object $X$, the morphism $\mathbf{Hom}(X, F) -> \mathbf{Hom}(U, F)$ obtained by precomposition is an isomorphism.
\end{definition}

A $j$-Sheaf is up to isomorphism the result of sheafifying an object.

$j$-Sheaves form a topos. The sheaves on a presheaf topos correspond to Beth semantics.

\subsection{Sheaves in type theory}

Consider a type theory with a notion of proof irrelevant propositions $\Prop$, \emph{e.g.} book-HoTT with mere propositions, or \textsc{Rocq} with $\SProp$.

In this case, a Lawvere-Tierney topology may be similarily defined, as a monad:
\begin{itemize}
    \item $\mathsf{J} : \Prop -> \Prop$
    \item $\eta : \Pi (P:\Prop).P -> \mathsf{J}\ P$
    \item $\bind : \forall (P Q: \Prop). \mathsf{J}\ P -> (P -> \mathsf{J}\ Q) -> \mathsf{J}\ Q$
\end{itemize}

Then a sheaf is just a type $T$ with 
\begin{itemize}
    \item A map $\ask_T : \Pi (P:\Prop).\ \mathsf{J}\ P -> (P -> T) -> T$
    \item A coherence $\varepsilon_T : \Pi (P:\Prop)\ (j : \mathsf{J}\ P)\ (x:T).\ \ask_T\ P\ j\ (\lambda p:P.x) = x$
\end{itemize}

Now, the sheafified of a type doesn't exists in general, if the theory admits quotient inductive types, it can then be defined as follow :

$$
\begin{array}{l}
    \Inductive\ \mathcal{S}_{\mathsf{J}}\ T : \Type := \\
    | \ret : T -> \mathcal{S}_{\mathsf{J}}\ T\\
    | \ask : \Pi (P:\Prop), \mathsf{J}\ P -> (P -> \mathcal{S}_{\mathsf{J}}\ T) -> \mathcal{S}_{\mathsf{J}}\ T\\
    |\ \varepsilon : \Pi (P:\Prop)\ (j : \mathsf J\ P)\ (x : \mathcal{S}_{\mathsf{J}}\ T). \ask\ P\ j\ (\lambda p:P. x) = x
\end{array}
$$

We note that by taking $I := \Sigma (P:\Prop). \mathsf{J}\ P$ and $O\ (P,j) : P$, a sheaf is then a
\begin{itemize}
    \item A type $T$
    \item A map $\ask_T : \Pi (i:I), (O\ i -> T) -> T$
    \item A coherence map $\varepsilon_T : \Pi(i:I)\ (x:T), \ask_T\ i\ (\lambda o:O\  i.x) = x$
\end{itemize}
and simimlarily for sheafification. This is marginally simpler, and make sheaves appear as quotient dialogue trees, hence why we will henceforth consider \emph{$(I,O)$-sheaves} instead of \emph{$\mathsf{J}$-sheaves}.

\section{Models}

A significant part of my internship was dedicated to contructing models of type theory in \textsc{Rocq}.

\begin{enumerate}
    \item A model of a variant of Baclofen TT using dialogue trees. Predicates must be linearized before eliminating an inductive into them.
    \item An exceptional model, with a type of exceptions $E$. A special type of dialogue trees where $I = E$ and $O i = \mathbf{0}$, the resulting theory is inconsistent (when $E$ is inhabited), as always when $O i -> \mathbf{0}$ for some $i$.
    \item A model using sheaves, which requires univalence, and quotient inductive types to model positive types.
\end{enumerate}


\section{System T and MLTT}
\subsection{MLTT}
We sought to extend the following variant of MLTT.

We consider a type of levels containing two elements $\slvl$ and $\llvl$ for small and large, with generic $\ell$

With terms :

$$ M,N ::= x | \lambda x. M | MN | 0 | S | \nat_{\rec} | \bot_{\rec} | \nat | \bot | \Pi x :A.B | \square_\slvl | \square_\llvl$$


Contexts :
$$ \Gamma ::= \Gamma, x : A | \cdot $$

And conversion rules :

$$
    \inference[\textsc{Wf-Empty}]{}{\cdot |- \Wf} \qquad
    \inference[\textsc{Wf-Ext}]{\Gamma |- A == A & \Gamma |- \Wf}{ \Gamma, x : A |- \Wf}
$$
$$
    \inference[\textsc{Int-Typ}]{\Gamma |- \Wf}{\Gamma |- \nat == \nat : \square_\ell} \qquad
    \inference[\textsc{Emp-Typ}]{\Gamma |- \Wf}{\Gamma |- \bot == \bot : \square_\ell}
$$
$$
    \inference[\textsc{Fun-Typ}]{\Gamma |- A == A' : \square_\ell & \Gamma, x : A |- B == B' : \square_\ell}{\Gamma |- \Pi x  : A, B == \Pi x: A', B' : \square_\ell} \qquad
    \inference[\textsc{Typ-Typ}]{\Gamma |- \Wf}{\Gamma |- \square_\slvl : \square_\llvl}
$$


$$
    \inference[\textsc{Fun-Intro}]{\Gamma, x : A |- M == M' : B &\Gamma |- A == A : \square_\ell}{\Gamma |- \lambda x. M == \lambda x. M' : \Pi x : A, B} \qquad
    \inference[\textsc{Fun-Elim}]{\Gamma |- M == M' : \Pi x:A, B& \Gamma |- N == N' : A}{\Gamma |- MN == M'N' : B(N/x)}
$$

$$
    \inference[\textsc{Axiom}]{\Gamma |- \Wf & x : A \in \Gamma }{\Gamma |- x == x : A} \qquad
    \inference[\textsc{Beta}]{\Gamma, x : A |- M == M' : B & \Gamma |- N == N' : A}{\Gamma |- (\lambda x. M)N == M'(N'/x) : B(N/x)}
$$

$$
    \inference[\textsc{Int-Zero}]{\Gamma |- \Wf}{\Gamma |- 0 == 0 : \nat} \qquad
    \inference[\textsc{Int-Succ}]{\Gamma |- \Wf}{\Gamma |- S == S : \nat -> \nat}
$$

$$
    \inference[\textsc{Int-Rec}]{\Gamma |- \Wf}{\Gamma |- \nat_{\rec} == \nat_{\rec} : \Pi A : \nat \to \square_\slvl, A 0 \to \left(\Pi n:\nat, A n \to A (S n)\right) \to \Pi n :\nat, A n}
$$


$$
    \inference[\textsc{Int-Rec-Zero}]{\Gamma |- A == A : \nat -> square_\slvl & \Gamma |- N_0 == N'_0 == A 0\\ \Gamma |- N_S == N_S : \Pi n:\nat, A n -> A(S n)}{\Gamma |- \nat_{\rec} A N_0N_S0 == N'_0 : A 0}
$$
$$
    \inference[\textsc{Int-Rec-Succ}]{\Gamma |- A == A' : \nat -> square_\slvl & \Gamma |- N_0 == N'_0 == A 0\\ \Gamma |- N_S == N'_S : \Pi n:\nat, A n -> A(S n) & \Gamma |- N == N' :\nat}{\Gamma |- \nat_{\rec} A N_0N_S (S N) == N'_S N' (\nat_{\rec}A' N'_0 N'_S N') : A (S N)}
$$


$$
    \inference[\textsc{Emp-Rec}]{\Gamma |- \Wf}{\Gamma |- \bot_{\rec} == \bot_{\rec} : \Pi A : \bot \to \square_\slvl,\Pi e :\bot, A e}
$$

$$
    \inference[\textsc{Sym}]{\Gamma |- M == M' : A}{\Gamma |- M' == M : A} \qquad
    \inference[\textsc{Trans}]
    {\Gamma |- M == M' :A & \Gamma |- M' == M'' : A}{\Gamma |- M == M'' : A}
$$

$$
    \inference[\textsc{Conv}]
    {\Gamma |- M == M' :A & \Gamma |- A == A' : \square_\ell}{\Gamma M ==M' : A'}
$$

\subsection{System T}

To identify and solve problems in a simpler envirronement, we studied a modified System T before, based on the following variant.

With types :

$$ A ::= A -> A | \nat | \bot $$

Terms :

$$ M,N ::= x | \lambda x. M | MN | 0 | S | \nat_{\rec} | \bot_{\rec} $$


Contexts :
$$ \Gamma ::= \Gamma, x : A | \cdot $$

And conversion rules :


$$
    \inference[\textsc{Fun-Intro}]{\Gamma, x : A |- M == M' : B }{\Gamma |- \lambda x. M == \lambda x. M' : A -> B} \qquad
    \inference[\textsc{Fun-Elim}]{\Gamma |- M == M' : A -> B& \Gamma |- N == N' : A}{\Gamma |- MN == M'N' : B}
$$

$$
    \inference[\textsc{Axiom}]{x : A \in \Gamma }{\Gamma |- x == x : A} \qquad
    \inference[\textsc{Beta}]{\Gamma, x : A |- M == M' : B & \Gamma |- N == N' : A}{\Gamma |- (\lambda x. M)N == M' : B}
$$

$$
    \inference[\textsc{Int-Zero}]{\Gamma |- \Wf}{\Gamma |- 0 == 0 : \nat} \qquad
    \inference[\textsc{Int-Succ}]{\Gamma |- \Wf}{\Gamma |- S == S : \nat -> \nat}
$$

$$
    \inference[\textsc{Int-Rec}]{\Gamma |- \Wf}{\Gamma |- \nat_{\rec} == \nat_{\rec} : A \to \left(\nat -> A -> A \right) -> \nat -> A}
$$


$$
    \inference[\textsc{Int-Rec-Zero}]{\Gamma |- N_0 == N'_0 == A\\ \Gamma |- N_S == N_S : \nat -> A  -> A}{\Gamma |- \nat_{\rec} N_0N_S0 == N'_0 : A}
$$
$$
    \inference[\textsc{Int-Rec-Succ}]{\Gamma |- N_0 == N'_0 == A\\ \Gamma |- N_S == N'_S : \nat -> A  -> A & \Gamma |- N == N' :\nat}{\Gamma |- \nat_{\rec} N_0N_S (S N) == N'_S N' (\nat_{\rec}N'_0 N'_S N') : A}
$$


$$
    \inference[\textsc{Emp-Rec}]{}{\Gamma |- \bot_{\rec} == \bot_{\rec} : \bot -> A}
$$

$$
    \inference[\textsc{Sym}]{\Gamma |- M == M' : A}{\Gamma |- M' == M : A} \qquad
    \inference[\textsc{Trans}]
    {\Gamma |- M == M' :A & \Gamma |- M' == M'' : A}{\Gamma |- M == M'' : A}
$$


\section{Meta-informations}

\subsection{Time expenditure}


\subsection{Difficulties}

The subject is rather vast and a bit unclear.

Doing logical relations for MLTT is a large task, even before adding sheaves. My next attempt will probably start from logrel rocq insted of nothing

%\subsection{Activities}

\section{Conclusion}



\newpage

\appendix
\appendixname

\section{Dummy}

\printbibliography
\end{document}